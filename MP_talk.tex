%
%  untitled
%
%  Created by Steven Harms (HOME) on 2011-07-26.
%  Copyright (c) 2011 Steven G. Harms. All rights reserved.
%
\documentclass[slidestop,compress,mathserif]{beamer}
\usepackage[bars]{beamerthemetree}
\usetheme{Antibes}
\usecolortheme{lily}

% Use utf-8 encoding for foreign characters
\usepackage[utf8]{inputenc}

% Surround parts of graphics with box
\usepackage{boxedminipage}

% Package for including code in the document
\usepackage{listings}

% This is now the recommended way for checking for PDFLaTeX:
\usepackage{ifpdf}

\ifpdf
\usepackage[pdftex]{graphicx}
\else
\usepackage{graphicx}
\fi
\title{A LaTeX Article}
\author{  }

\date{2011-07-26}

\begin{document}

\ifpdf
\DeclareGraphicsExtensions{.pdf, .jpg, .tif}
\else
\DeclareGraphicsExtensions{.eps, .jpg}
\fi

\maketitle


\begin{abstract}
\end{abstract}

\section{Introduction}

\bibliographystyle{plain}
\bibliography{}
\end{document}

-------


------------------------------------------------------------------------------
------------------------------------------------------------------------------


[SLIDE]
Practical Metaprogramming:  Modeling Thought


Steven Harms
Date:  August 12, 2011

Good morning, I want to welcome you all to the first day of Lone Star Ruby
Conference V. It's great to be back here in Austin. I've lived here for over
eight years off and on and love this town, its vibrant Ruby community, and
creativity with turning cheese into a soup that you put chips in. If you don't
know what I'm talking about, talk to me later and we'll get you set up
properly.

[SLIDE]

Contact:  Steven Harms
Twitter:  @sgharms
GitHub:  sgharms
G+
http://stevengharms.com

These are my contact details, if you would like them here's your chance to
write them down. If you have questions or recommendations about this
presentation, feel free to contat me via one of these methods.

[SLIDE]

What we are here to talk about today should be:

[SLIDE]

Practical Metaprogramming

or

[SLIDE]

Modeling Thought

or

[SLIDE]

``Things I learned while modeling a 2,500 year old dead language in Ruby''

Now I'll assume that as I have only met one other person interested in
modeling languages in Ruby in 5 years of hanging out amongst you fine folks,
I'm going to assume you're not going to talk about this one.

[SLIDE]

@dmiessler: ``Is there anything from your presentation that you left out of
that title?''

[SLIDE]

I'd like to give you a bit of an overview of where we're going to go today.  

[SLIDE]

Practical Metaprogramming: 

This describes content that will help developers
familiar with basic Metaprogramming techniques level-up in Ruby. I'll also
cover why you \emph{need} to understand Ruby's MP model. I'll also talk a bit
about the FUD around whether or not you should make heavy use of MP.

[SLIDE]

I hope that after demonstrating why you need to understand MP and why you can't
avoid it, you will then be curious about how to decide where to use
Metaprogramming. I will argue that your guide in this should be ``modeling
thought.'' This is a term of my own invention, so don't look that up in your
Pickaxe book. 

[SLIDE]

What I mean by this term is not some sort of scary AI vision for
the future, involving humans as batteries 

[SLIDE]

Or killer cyborgs with gubenatorial careers. Rather I can sum it up as: 

[SLIDE]

``When you seek to model problem domains where method calls may be ambiguous
bu where you demand discrete return sets.''

[SLIDE]

A running thread though this will be my code project LatinVerb which is a
library used to conjugate all 5 paradigms of Verbs in Latin into the 45-odd
tenses and near 200 unique forms \emph{of each verb}. It makes heavy use of
the ``modeling thought'' paradigm but was also a project that would have been
horribly unweildy \emph{had it not been} for Ruby's metaprogramming
capaabilities.

[SLIDE]  

That's the lay of the land for this talk. If you're thinking this talk might
not be for you, then you're welcome to hop over to Wynn's talk to learn about
mobile development and deployment using TI.

[SLIDE:  I lied guy]

``I lied''

[SLIDE:  SAP]

Whew, I'm glad we only lost those few. Honestly I was afraid that that gesture
of open communication might backfire

[SLIDE:  Forever Alone]

As I said in the beginning, our first topic to undertake is ``Practical''
metaprogramming. Can I get a show of hands, how many of you were lured to Ruby
by its MP capabilities?  

I'll admit it, when I saw this example, I was pretty wowed.

\begin{verbatim}
	ruby-1.9.2-p180 > "pee-wee".respond_to? :awake
	 => false 
\end{verbatim}

\begin{verbatim}
	ruby-1.9.2-p180 > class String
	ruby-1.9.2-p180 ?>  def awake; 'I remember...The Alamo!'; end
	ruby-1.9.2-p180 ?>  end
	 => nil 
	ruby-1.9.2-p180 > "pee-wee".respond_to? :awake
	 => true 
	ruby-1.9.2-p180 > "pee-wee".awake
	 => "I remember...The Alamo!"
\end{verbatim}

Or consider what we find in the earliest pages of the Pickaxe:

\begin{verbatim}
	class Song
	  attr_reader :name, :artist, :duration
	end
\end{verbatim}

\underline{Programming Ruby}.  8th ed., 2006.